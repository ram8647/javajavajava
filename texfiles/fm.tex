%FM Morelli
\documentclass{book}
%
%
\input morelli.sty
\input morelliFM.sty
%
%
\raggedbottom%��Eliminata giustif. verticale, mettere % prima di stampare
%
%
\begin{document}
%
\setcounter{page}{1}
\pagenumbering{roman}
%
\spnorm\thispagestyle{empty}
\parindent 0pc
\begin{minipage}{40.5pc}
\begin{center}
\vspace*{70pt}\fontFMtped SECOND EDITION

\vspace*{36pt}{\color{cyan}\fontFMtpt JAVA, JAVA, JAVA\raisebox{35pt}{\fontFMtptsup TM}!}

\vspace*{12pt}\fontFMtpst Object-Oriented Problem Solving

\vspace*{48pt}{\color{cyan}\fontFMtpau Ralph Morelli}

\vspace*{6pt}\fontFMtpaf Trinity College

\vspace*{40pt}{\color{cyan}\fontFMtpaf Alan Apt Series}

\vspace*{144pt}\epsfig{file=e2kl-D1:PH-e2kl:Morelli:0333700MOREL:FM:prenhall_b&w.eps,width=4pc,clip=}

\vspace*{3pt}\bf Prentice Hall

\rm Upper Saddle River, New Jersey 07458


\end{center}
    \begin{textblock}{1}(-3,-3)%
    \epsfig{file=e2kl-D1:PH-e2kl:Morelli:0333700MOREL:commonart:morelli.eps}%
    \end{textblock}%
    \begin{textblock}{1}(0,17)%
    \epsfig{file=e2kl-D1:PH-e2kl:Morelli:0333700MOREL:FM:FM_CO1.eps}%
    \end{textblock}%
    \begin{textblock}{1}(40,17)%
    \epsfig{file=e2kl-D1:PH-e2kl:Morelli:0333700MOREL:FM:FM_CO2.eps}%
    \end{textblock}%
    \begin{textblock}{1}(0,-.75)%
    \epsfig{file=e2kl-D1:PH-e2kl:Morelli:0333700MOREL:FM:TitleArt_cyan.eps}%
    \end{textblock}%
\end{minipage}

\clearpage
\thispagestyle{empty}
    \begin{textblock}{1}(-3,-3)%
    \epsfig{file=e2kl-D1:PH-e2kl:Morelli:0333700MOREL:commonart:morelli.eps}%
    \end{textblock}%

\small\baselineskip11pt{\fontFMcip Library of Cataloging-in-Publication Data}

\vspace{5pt}Morelli, R. (Ralph)

\qquad Java, Java, Java : object oriented problem solving/by Ralph Morelli.---2nd ed.

\qquad \quad p.\quad cm.

\qquad ISBN 0-13-033370-0

\qquad 1. Object-oriented programming (Computer science) 2. Java (Computer 

\quad program language) I. Title.

\quad QA76.64 .M64 2002

\quad 005.13'3---dc21\hfill 2001050018



\vspace{32pt}Vice President and Editorial Director: {\it Marcia Horton}

Publisher: {\it Alan Apt}

Associate Editor: {\it Toni D. Holm}

Editorial Assistant: {\it Patrick Lindner}

Vice President and Director of Production 

\hspace*{1pc}and Manufacturing, ESM: {\it David W. Riccardi}

Executive Managing Editor: {\it Vince O'Brien}

Assistant Managing Editor: {\it Camille Trentacoste}

Developmental Editor: {\it Jerry Ralya}

Production Editor: {\it Fran Daniele}

Director of Creative Services: {\it Paul Belfanti}

%���Senior Manager, Artworks: {\it Patricia Burns}

Creative Director: {\it Carole Anson}

Art Director: {\it Heather Scott}

Art Editor: {\it Xiahong Zhu}

Design Technical Support: {\it John Christiana}

Manufacturing Manager: {\it Trudy Pisciotti}

Manufacturing Buyer: {\it Lynda Castillo}

Marketing Assistant: {\it Barrie Rheinhold}



%���Production/Composition Services: {\it Prepar\'{e}, Inc.}


\vspace{17pt}\epsfig{file=e2kl-D1:PH-e2kl:Morelli:0333700MOREL:FM:prenhall_b&w.eps,width=3pc,clip=}

\vspace{-25pt}\hspace*{3.5pc}\copyright\ 2003 by Prentice-Hall, Inc.%���Pearson Education

\hspace*{3.5pc}Upper Saddle River, New Jersey 07458


\vspace{17pt}All rights reserved. No part of this book may be reproduced, in any form 

or by any means, without permission in writing from the publisher.


\vspace{22pt}Printed in the United States of America

10\quad 9\quad 8\quad 7\quad 6\quad 5\quad 4\quad 3\quad 2\quad 1

\vspace{17pt}{\fontFMocra ISBN 0-13-033370-0}

\vspace{17pt}Pearson Education Ltd.,  {\it London}

Pearson Education Australia Pty. Ltd.,  {\it Sydney}

Pearson Education Singapore, Pte. Ltd.

Pearson Education North Asia Ltd.,  {\it Hong Kong}

Pearson Education Canada, Inc.,  {\it Toronto}

Pearson Educac\'{\i}on de Mexico, S.A. de C.V.

Pearson Education---Japan, {\it Tokyo}

Pearson Education Malaysia, Pte. Ltd.

Pearson Education, {\it Upper Saddle River, New Jersey}

\clearpage
\thispagestyle{empty}

\vspace*{132pt}{\fontFMded\color{cyan}To My Parents, Ralph and Ann Morelli}

    \begin{textblock}{1}(29,16)%
    \epsfig{file=e2kl-D1:PH-e2kl:Morelli:0333700MOREL:FM:FM_CO3.eps}%
    \end{textblock}%
    \begin{textblock}{1}(-3,-3)%
    \epsfig{file=e2kl-D1:PH-e2kl:Morelli:0333700MOREL:commonart:morelli.eps}%
    \end{textblock}%


\clearpage

\thispagestyle{plain}
\mbox{ }

    \begin{textblock}{1}(-.75,-.75)%
    \epsfig{file=e2kl-D1:PH-e2kl:Morelli:0333700MOREL:FM:Morelli_walk_01.eps}%
    \end{textblock}%

    \begin{textblock}{1}(-3,-3)%
    \epsfig{file=e2kl-D1:PH-e2kl:Morelli:0333700MOREL:commonart:morelli.eps}%
    \end{textblock}%

\clearpage

\mbox{ }

\thispagestyle{plain}

    \begin{textblock}{1}(-0.75,-.75)%
    \epsfig{file=e2kl-D1:PH-e2kl:Morelli:0333700MOREL:FM:Morelli_walk_02.eps}%
    \end{textblock}%

    \begin{textblock}{1}(-3,-3)%
    \epsfig{file=e2kl-D1:PH-e2kl:Morelli:0333700MOREL:commonart:morelli.eps}%
    \end{textblock}%

\clearpage

\thispagestyle{plain}
\mbox{ }


    \begin{textblock}{1}(-3,-3)%
    \epsfig{file=e2kl-D1:PH-e2kl:Morelli:0333700MOREL:commonart:morelli.eps}%
    \end{textblock}%

    \begin{textblock}{1}(-0.75,-.75)%
    \epsfig{file=e2kl-D1:PH-e2kl:Morelli:0333700MOREL:FM:Morelli_walk_03.eps}%
    \end{textblock}%

\clearpage

\thispagestyle{plain}
\mbox{ }


    \begin{textblock}{1}(-3,-3)%
    \epsfig{file=e2kl-D1:PH-e2kl:Morelli:0333700MOREL:commonart:morelli.eps}%
    \end{textblock}%

    \begin{textblock}{1}(-0.75,-.75)%
    \epsfig{file=e2kl-D1:PH-e2kl:Morelli:0333700MOREL:FM:Morelli_walk_04.eps}%
    \end{textblock}%

\clearpage

\thispagestyle{empty}
\mbox{ }
    \begin{textblock}{1}(-3,-3)%
    \epsfig{file=e2kl-D1:PH-e2kl:Morelli:0333700MOREL:commonart:morelli.eps}%
    \end{textblock}%

\clearpage
\thispagestyle{plain}
\normalsize\parindent 1pc
\setcounter{table}{0}
\setcounter{figure}{0}
\renewcommand{\thetable}{\mbox{\arabic{table}}}%
\renewcommand{\thefigure}{\mbox{\arabic{figure}}}%
\markboth{\fontRHch\color{cyan}PREFACE\hspace*{5pt}$\bullet$}{\color{cyan}$\bullet$\hspace*{5pt}\fontRHch PREFACE}
\vspace*{60pt}\noindent{\fontFMprt\color{cyan} Preface}
    \begin{textblock}{1}(11,10.5)%
    \noindent\epsfig{file=e2kl-D1:PH-e2kl:Morelli:0333700MOREL:FM:FM_CO4.eps}%
    \end{textblock}%
    \begin{textblock}{1}(-3,-3)%
    \noindent\epsfig{file=e2kl-D1:PH-e2kl:Morelli:0333700MOREL:commonart:morelli.eps}%
    \end{textblock}%
\vspace{10pt}\section*{Who Should Use This Book?}
\noindent The topics covered and the approach taken in this book are suitable for
a typical Introduction to Computer Science (CS1)
course or for a slightly more advanced Java as a Second Language
course.   The book is also useful to professional programmers
making the transition to Java and object-oriented programming.

The book takes an ``objects first'' approach to programming and
problem solving.  It assumes no previous programming experience and
requires no prior knowledge of Java or object-oriented programming.

\section*{What's New in the Second Edition?}
\noindent The second edition has the following substantive changes:

\begin{BL}
\item {\bf Unified Modeling Language (UML).} UML diagrams have been
incorporated throughout the text to help illustrate object-oriented
concepts and to describe the design of the Java programs we
develop. Some of the UML diagrams replace diagrams used in the first
edition, but many new diagrams have been introduced as well.

\noindent\hspace*{1pc}UML is rapidly developing into an industry standard for designing
object-oriented programs. So it will be useful for students to be
familiar with it.  But the main reason for incorporating UML is a
pedagogical one: It makes it easier to present and teach
object-oriented concepts such as information hiding, inheritance, and
polymorphism.

\item {\bf Emphasis on Object-Oriented Design.} The second edition
places somewhat more emphasis on object-oriented design, mostly
through the way examples are developed. Many of the programming
examples were rewritten to focus on the design before getting into the
Java coding details. This approach was greatly aided by the use of UML
diagrams.

\item {\bf Organizational Changes.} Based on suggestions from reviewers
and users of the first edition, several chapters were reorganized. The
first two chapters were extensively revised. Chapter 0 now provides a
more detailed overview of object orientation and introduces the main
features of UML. This should help orient the student before getting
into the discussion of Java's implementation of object-oriented concepts.
To underscore the
importance of design versus coding,  Chapter 1 has been reorganized 
to focus on program design and
development before introducing any Java code. It also includes a concise overview
of some of the basic Java language features.  Thus, the
student has been presented with a good bit of overview material
before we begin looking at specific Java examples.

\noindent In terms of the organization of Java language elements, discussion of
the {\tt switch} statement was moved to Chapter 3, where it is covered
along with the other selection-control structures. Chapter 11
``Exceptions'' was reorganized so that it can more easily be used much
earlier in the course.

\end{BL}

\section*{Why Start with Objects?}
\noindent {\it Java, Java, Java} takes an ``objects early'' approach to teaching
Java, with the assumption that teaching beginners the ``big picture''
early gives them more time to master the principles of object-oriented
programming.

The first time I taught Java in our CS1 course, I followed the same
approach I had been taking in teaching C and C++---namely, start
with the basic language features and structured programming concepts
and then, somewhere around midterm, introduce object orientation.
This approach was familiar, for it was one taken in most of
the textbooks then available in both Java and C++.

One problem with this approach was that many students failed to get
the big picture.   They could understand loops, if-else constructs,
and arithmetic expressions, but they had difficulty decomposing a
programming problem into a well-organized Java program.  Also, it
seemed that this procedural approach failed to take advantage of the
strengths of Java's object orientation.  Why teach an object-oriented
language if you're going to treat it like C or Pascal?

I was reminded of a similar situation that existed when Pascal was the
predominant CS1 language.  Back then the main hurdle for beginners was
{\it procedural abstraction}---learning the basic mechanisms
of procedure call and parameter passing and learning how to {\bf
design} programs as a collection of procedures.  {\it Oh! Pascal!},
my favorite introductory text, was typical of a ``procedures early''
approach.  It covered procedures and parameters in Chapter 2, right
after covering the assignment and I/O constructs in Chapter 1. It then
covered program design and organization in Chapter 3. It didn't get
into loops, if-else, and other structured programming concepts until
Chapter 4 and beyond.

Presently, the main hurdle for beginners is {\it object
abstraction}. Beginning programmers must be able to see a program as a
collection of interacting objects and must learn how to decompose
programming problems into well-designed objects.   Object orientation
subsumes both procedural abstraction and structured programming
concepts from the Pascal days.   Teaching ``objects early'' takes a
top-down approach to these three important concepts.   The sooner you
begin to introduce objects and classes, the better the chances that
students will master the important principles of object orientation.

Object orientation (OO) is a fundamental problem-solving and design
concept, not just another language detail that should be relegated to
the middle or the end of the book (or course).  If OO concepts are
introduced late, it is much too easy to skip over them when push comes
to shove in the course.

Java is a good language for introducing object orientation.  Its object
model is better organized than C++. In C++, it is easy to ``work
around'' or completely ignore OO features and treat the language like
C.~In Java, there are good opportunities for motivating the discussion
of object orientation.  For example, it's almost impossible to discuss
applets without discussing inheritance and polymorphism.  Thus, rather
than using contrived examples of OO concepts, instructors can use some
of Java's basic features---applets, the class library, GUI
components---to motivate these discussions in a natural way.

\vspace{7pt}\section*{Key Features}
\noindent In addition to its ``objects early'' approach, this book has several
other important features.
\begin{BL}

\item {\bf Unified Modeling Language (UML) Diagrams.} More than 225
UML diagrams have been incorporated throughout the text to explain
object-oriented concepts and to focus on object-oriented design.  The
advantages of using UML are several. First, UML diagrams provide a
concise visual means of describing the main features of classes and
objects. In one glance you can see an object's attributes and methods,
whether they are private or public, and how the class or object
relates to other classes. Second, UML diagrams provide simple
graphical models for important object-oriented concepts such as
inheritance and polymorphism. A picture is worth 1,000 words, so in
addition to the descriptions provided in words, the UML diagrams
should help students learn these important concepts. Third, for the
purposes for which it is used in this book, UML is relatively easy and
intuitive to understand. The basic notation is introduced in Chapter 0
through simple, accessible examples. Finally, UML is rapidly becoming
an industry standard. So gaining familiarity with UML in this book
will make other books on Java or object-oriented design more
accessible.

\item  {\bf The {\tt CyberPet} Example.} Throughout the text a {\tt
CyberPet} class is used as a running example to motivate and
illustrate important concepts.  The {\tt CyberPet} is introduced in
Chapter 2, as a way of ``anthropomorphizing'' the basic features of
objects.   Thus, individual {\tt CyberPet}s belong to a class
(definition), have a certain state (instance variables), and are
capable of certain behaviors like eating and sleeping (instance
methods). Method calls are used to command the {\tt CyberPet}s to
eat and sleep.   In Chapter 3 the emphasis is on defining and using
methods and parameters to promote communication with {\tt CyberPet}s.
In subsequent chapters, concepts such as inheritance, randomness,
animation, and threads are illustrated in terms of the {\tt CyberPet}.
Some of the lab and programming exercises are also centered around
extending the behavior and sophistication of the {\tt CyberPet}.

\item  {\bf Applets and GUIs.} Applets and GUIs are first introduced in
Chapter 4 and then used throughout the rest of the text.  Clearly,
applets are a ``turn-on'' for introductory students and can be used as
a good motivating factor.  Plus, {\it event-driven programming} and
Graphical User Interfaces (GUIs) are what students ought now to be
learning in CS1. We are long past the days when command-line
interfaces were the norm in applications programming.   Another nice
thing about Java applets is that they are fundamentally
object oriented.  To understand them fully, students need to understand
basic OO concepts.   That's why applets are not introduced until
Chapter 4, where they provide an excellent way to motivate the
discussion of inheritance and polymorphism.

\vspace{3pt}\item  {\bf Companion Web Site.} The text is designed to be used in
conjunction with a companion Web site that includes many useful
resources, including the Java code and Java documentation (in HTML) for
all the examples in the text, additional lab and programming
assignments, online quizzes that can be scored automatically,
and PowerPoint class notes.

\vspace{3pt}\item  {\bf Problem-Solving Approach.} A pedagogical, problem-solving
approach is taken throughout the text.  There are a total of 13 fully
developed case studies, as well as numerous other examples that
illustrate the problem-solving process.   Marginal notes in the text
repeatedly emphasize the basic elements of object-oriented problem
solving: What objects do we need? What methods and data do we need?
What algorithm should we use? And so on.

\vspace{3pt}\item  {\bf Self-Study Exercises.} The book contains more than 200
self-study exercises, with answers provided at the back of each
chapter.

\vspace{3pt}\item  {\bf End-of-Chapter Exercises.} Over 400 end-of-chapter exercises
are provided, including ``Challenge'' exercises at the end of most
sets.  The answers are provided in an Instructor's Manual, which is available
to adopters.

\vspace{3pt}\item  {\bf Programming, Debugging, and Design Tips.} The book contains
nearly 400 separately identified ``tips'' (Programming Tips, Debugging
Tips, Effective Design Principles, and Java Language Rules) that
provide useful programming and design information in a
nutshell.

\vspace{3pt}\item  {\bf Hands-On Learning Sections.} Each chapter concludes with a
laboratory exercise, so the text can easily be used to support
lab-based CS1 courses. For CS1 courses that are not
lab based, these sections can still be read as preparation for a
programming assignment, or as an in-class demo, or as some other form of 
hands-on exercise. For each lab in the text, the companion Web
site contains additional resources and handouts, as well as a
repository of alternative lab assignments.

\vspace{3pt}\item  {\bf ``From the Library'' Sections.} Each chapter includes a section
that introduces one or more of the library classes from the Java API
(Application Programming Interface).  In the early chapters, these
sections provide a way of introducing tools, such as I/O classes and
methods, needed to write simple programs.  In subsequent chapters, some
of these sections introduce useful but optional topics, such as the
{\tt NumberFormat} class used to format numeric output.  Others introduce
basic GUI (Graphical User Interface) components that are used in
program examples and the laboratory sections.

\vspace{3pt}\item  {\bf ``Object-Oriented Design'' Sections.} Each chapter includes a
section on object-oriented design, which is used to underscore and
amplify important principles such as inheritance, polymorphism, and
information hiding. For instructors wishing to emphasize
object-oriented design, Table~1 provides a list of
sections that should be covered.

\vspace{3pt}\item  {\bf ``Java Language Summary''.} Those chapters that introduce
language features contain ``Java Language Summary'' sections that
summarize the features' essential syntax and semantics.
\end{BL}

\begin{table}[tb]
%\hphantom{\caption{An overview of object-oriented sections.}\label{tab-ood}}
\TBT{0.5pc}{An overview of object-oriented sections.}
\begin{tabular}{ll} 
\multicolumn{2}{l}{\color{cyan}\rule{26pc}{1pt}}\\[2pt]
\TBCH{Topic}                  & \TBCH{Section} 
\\[-4pt]\multicolumn{2}{l}{\color{cyan}\rule{26pc}{0.5pt}}\\[2pt]
What Is Object Orientation?                       &  Chapter 0.7\\
Overview of UML                                  &  Chapter 0.8\\
Object-Oriented Design Process                   &  Chapter 1.2\\
Objects: Defining, Creating, Using               &  Chapter 2\\
Methods: Communicating with Objects              &  Chapter 3\\
Inheritance: The {\tt toString()} Method         & Chapter 3, OOD\\
Inheritance and Polymorphism in Applets          & Chapter 4.2-4.4\\
Inheritance and Polymorphism: {\tt ToggleButton} & Chapter 4, OOD\\
Information Hiding                               & Chapter 5, OOD\\
Structured Programming Concepts                  & Chapter 6, OOD\\
Abstract Classes: {\tt Cipher}                   & Chapter 7, OOD\\
Polymorphism: Polymorphic Sorting                & Chapter 8, OOD\\
Model-View-Controller Architecture               & Chapter 9, OOD\\
Inheritance and Polymorphism: Spider/Fly Classes \hspace*{1pc}& Chapter 13.6\\
Generic Client/Server Classes                    & Chapter 15.7\\
Abstract Data Types: {\tt List} Class            & Chapter 16, OOD
\\[-4pt]\multicolumn{2}{l}{\color{cyan}\rule{26pc}{1pt}}
\end{tabular}
\endTB
\vspace{-9pt}\end{table}

\section*{Organization of the Text}
\noindent The book is organized into three main parts.  The first part (Chapters
0 through 4) introduces the basic concepts of object orientation,
including objects, classes, methods, parameter passing, information
hiding, inheritance, and polymorphism.   Although the primary focus in
these chapters is on object orientation rather than on Java language
details, each of these chapters has a ``Java Language Summary'' section
that summarizes the language elements introduced.

In Chapters 1 to 3, students are given the basic building blocks for
constructing a Java program from scratch.  Although the programs at
this stage have limited functionality in terms of control structures
and data types, the priority is placed on how objects are constructed
and how they interact with each other through method calls and
parameter passing.

The second part (Chapters 5 through 8) focuses on the remaining
language elements, including data types and operators (Chapter 5),
control structures (Chapter 6), strings (Chapter 7), and arrays
(Chapter 8).  Once the basic structure and framework of an
object-oriented program are understood, it is relatively easy to
introduce these language features.

Part Three (Chapters 9 through 16) covers a variety of advanced
topics, including Graphical User Interfaces (Chapter 9), graphics
(Chapter 10), exceptions (Chapter 11), recursion (Chapter 12), threads
(Chapter 13), files (Chapter 14), sockets (Chapter 15), and data
structures (Chapter 16). Topics from these chapters can be used
selectively depending on instructor and student interest.

Table 2 provides an example syllabus from my %<tab_syllabus1>
one-semester CS1 course.  Our semester is 13 weeks (plus one reading
week during which classes do not meet).


\splarge Note that the advanced topic chapters needn't be covered in
order.  Recursion (Chapter 12) could be introduced at the same time or
even before loops (Chapter 6). The recursion chapter includes examples
using strings, arrays,  and  drawing  algorithms  (fractals),  \spnormallar
\begin{table}[htb]%
%\hphantom{\caption{A one-semester course.}\label{tab-course}}
\TBT{3.5pc}{A one-semester course.}
\hspace*{3pc}\begin{tabular}{lll}
\multicolumn{3}{l}{\color{cyan}\rule{20pc}{1pt}}\\[2pt]
\TBCH{Weeks} & \TBCH{Topics}                  & \TBCH{Chapters} 
\\[-4pt]\multicolumn{3}{l}{\color{cyan}\rule{20pc}{0.5pt}}\\[2pt]
   1           & Object Orientation, UML         &  Chapter 0\\
               & Program Design and Development  &  Chapter 1\\
   2--4        & Objects and Class Definitions &  Chapter 2  \\
               & Methods and Parameters        &  Chapter 3  \\
               & Selection Structure (if-else) &             \\
   5           & Applet Programming            &  Chapter 4  \\
               & Inheritance                   &             \\
   6           & Data Types and Operators      &  Chapter 5  \\
   7--8        & Control Structures (Loops)    &  Chapter 6  \\
               & Structured Programming        &             \\
   9           & String Processing (loops)     &  Chapter 7  \\
   10          & Array Processing              &  Chapter 8  \\
   11          & Recursion                     &  Chapter 12 \\
   12          & Advanced Topic (GUIs)         &  Chapter 9  \\
   13          & Advanced Topic (Exceptions)   &  Chapter 11 \\
               & Advanced Topic (Threads)      &  Chapter 13     
\\[-4pt]\multicolumn{3}{l}{\color{cyan}\rule{20pc}{1pt}}
\end{tabular}
\endTB
\vspace{-6pt}\end{table}%
as  well  as some standard numerical algorithms (factorial). Another way to teach
recursion would be to incorporate it into the discussion of strings
(Chapter 7), arrays (Chapter~8), and graphics (Chapter 10), thereby
treating iteration and recursion in parallel.

Exceptions (Chapter 11) could also be covered earlier.  The examples in
the first few sections of this chapter use simple arithmetic
operations and the basic for loop.  If these language elements are
introduced separately, then exceptions could be covered right after
Chapter 3.


Some of the examples in the advanced chapters use applets (Chapter~4)
and GUIs (Chapter 9), so these chapters should ideally be covered
before Chapters 10 (graphics), 13 (threads), 14 (files), and 15
(sockets and networking).  However, Chapter 16 (data structures) and
sections of the other advanced topic chapters can be covered
independently of applets and GUIs.  Figure~\ref{fig-depend} shows the
major chapter dependencies in %<fig_dependency> 
the book.

%\begin{figure}
\begin{figure}[h]
\figavarleft{FM:preface.eps}{20.25pc}{Chapter dependencies. 
\label{fig-depend}}
\end{figure}
%\end{figure}







\section*{Acknowledgments}
\noindent First, I would like to thank the reviewers and technical reviewers,
whose comments often suggested important additions and revisions.  For
the first edition these include Pedro Larios (Metrowerks), Laird
Dornin (Sun Microsystems), Katherine Lowrie (Trilogy Inc.), Robert
Holloway (University of Wisconsin at Madison), Deborah Trytten
(University of Oklahoma), Alan Miller (Golden Gate University), Haklin
Kimm (University of Tennessee), and Jim Roberts (Carnegie Mellon
University).

For the second edition the reviewers include Jim Buffenbarger (Idaho
State University), Dianne Wolff (Western Virginia Community College),
Hamid Namati (University of North Carolina at Greensboro), Kelli Davis
(Cape Fear Community College, North Carolina), Le Gruenwald
(University of Oklahoma), Haklin Kimm (East Stroudsburg University),
Dave Barrington (University of Massachusetts at Amherst), and John
Ellis (University of Wyoming). The following UML experts deserve
thanks for their many helpful suggestions regarding the use of UML in the
book: William Tepfenhart (Monmouth University), Rolf Kamp (AT\&T),
Scott Henninger (Univerity of Maryland), and Michael Huhns (University
of South \mbox{Carolina).}

At Prentice Hall, my appreciation goes to Sandi Hakansan for
encouraging me to send the manuscript to Prentice Hall; to Alan Apt for taking on
this project and for being a source of guidance and encouragement
throughout; to Ana Arias Terry for her cheerful and professional
management of the project; to Rose Kernan for guiding the project
through a very tough production phase; and to Eric Unhjem and Toni Holm for
their help and support.  I especially want to thank my development
editor, Jerry Ralya, whose careful work and many suggestions helped
shape and improve the final result of both editions in immeasurable ways.

I want to thank a number of Trinity College students who helped track down
typos and other errors in earlier drafts of the text.  These include
Christian Allen, Jeff Green, Ryan Carmody, and Michael Wilson.  I especially
want to thank Jamie Mazur, whose feedback on the manuscript and enthusiasm
for the objects-first approach and whose role in developing the solutions
manual were much appreciated. My thanks to the following list of individuals who 
helped identify errors in the first edition: Ashton Hobbs, Elissa Lowe, Peter 
Casey, Bernd Bruegge, David Mix Barrington,
Rommy Mayerowitz, Jason Smart, Colleen Kubont, Jonathan Amery, Shakira Ramos, 
Jennifer Chen, Robert Gaspar, Yolanda McManus, and
Jim Parry.

Thanks to my colleague Joe Palladino in the Engineering Department,
who served as the La\TeX\ meister throughout the project; to Chuck
Liang in the Computer Science Department for interesting Java
examples; and to my good Hawaii friend and limerick meister, Lanning
Lee.

To my Trinity computer science colleagues, Madalene Spezialetti and
Ralph Walde, thanks for their generous support and advice, especially
during the early stages of this project.  They're the ones who pulled
me back over the rail when I went overboard with the ``objects first''
juggernaut.  They also helped rescue CyberPet from its germ as a
(lame!) horse.  With their conjurings and refinements it evolved
through a dog, a parrot, and then a ``net pet,'' before settling,
rather nicely we think, as CyberPet.  They also have my gratitude for
being willing to subject our students to earlier (incomplete) drafts
of the manuscript in their sections of our intro course.  Madalene's
suggestion to emphasize basic language features early led eventually
to the book's ``Java Language Summary'' sections.  She also took the
first pass at drafting material that appears in Chapter 0 and 1.
Although that material has been rewritten (and whatever flaws it has
are entirely mine), it still bears the stamp of her influence.  Their
criticisms and suggestions have improved the text in immeasurable and
significant ways, and the gentleness, subtleness, and humor with which
they delivered their suggestions have helped sustain our friendship.

Finally, thanks to my wife, Choong Lan How, for her love and
encouragement, and for her careful reading of the first three
chapters; to my daughter Alicia for her feedback on artistic matters;
and to my daughter Meisha for her tremendous help in producing the
solutions manual and the on-line study guide.



\clearpage

\thispagestyle{plain}
\markboth{\fontRHch\color{cyan}Brief Contents\hspace*{5pt}$\bullet$}{\color{cyan}$\bullet$\hspace*{5pt}\fontRHch Brief Contents}
    \begin{textblock}{1}(-3,-3)%
    \noindent\epsfig{file=e2kl-D1:PH-e2kl:Morelli:0333700MOREL:commonart:morelli.eps}%
    \end{textblock}%

\vspace*{60pt}\noindent{\fontFMprt\color{cyan} Brief Contents}
    \begin{textblock}{1}(15,10.5)%
    \noindent\epsfig{file=e2kl-D1:PH-e2kl:Morelli:0333700MOREL:FM:FM_CO3.eps}%
    \end{textblock}%
\vspace*{26pt}\fontFMbrftcct\baselineskip16pt

\begin{BCch}
\addtocounter{BCchcount}{-1}
\item[] Preface \hfill ix
\item Computers, Objects, and Java\hfill 3
\item Java Program Design and Development\hfill 29
\item Objects: Defining, Creating, and Using\hfill 67
\item Methods: Communicating with Objects\hfill 115
\item Applets: Programming for the World Wide Web\hfill 167
\item Java Data and Operators\hfill 219
\item Control Structures\hfill 275
\item Strings and String Processing\hfill 331
\item Arrays and Array Processing\hfill 377
\item Graphical User Interfaces\hfill 439
\item Graphics and Drawing\hfill 495
\item Exceptions: When Things Go Wrong\hfill 551
\item Recursive Problem Solving\hfill 595
\item Threads and Concurrent Programming\hfill 635
\item Files, Streams, and Input/Output Techniques\hfill 691
\item Sockets and Networking\hfill 737
\item Data Structures: Lists, Stacks, and Queues\hfill 791
\end{BCch}
\begin{BCap}
\item Coding Conventions\hfill 824
\item The Java Development Kit\hfill 831
\item The ASCII and Unicode Character Sets\hfill 839
\item Java Keywords\hfill 840
\item Operator Precedence Hierarchy\hfill 841
\item Advanced Language Features\hfill 843
\item Java and UML Resources\hfill 849
\item[] Subject Index\hfill 851
\end{BCap}


\clearpage
\mbox{ }
\thispagestyle{empty}
    \begin{textblock}{1}(-3,-3)%
    \noindent\epsfig{file=e2kl-D1:PH-e2kl:Morelli:0333700MOREL:commonart:morelli.eps}%
    \end{textblock}%

\clearpage

\thispagestyle{plain}
\markboth{\fontRHch\color{cyan}CONTENTS\hspace*{5pt}$\bullet$}{\color{cyan}$\bullet$\hspace*{5pt}\fontRHch CONTENTS}
    \begin{textblock}{1}(-3,-3)%
    \noindent\epsfig{file=e2kl-D1:PH-e2kl:Morelli:0333700MOREL:commonart:morelli.eps}%
    \end{textblock}%

\vspace*{60pt}\noindent{\fontFMprt\color{cyan}Contents}
    \begin{textblock}{1}(11,10.5)%
    \noindent\epsfig{file=e2kl-D1:PH-e2kl:Morelli:0333700MOREL:FM:FM_CO4.eps}%
    \end{textblock}%
\vspace*{26pt}\normalsize\baselineskip12pt



\begin{CH}
\addtocounter{TOCchcount}{-1}
\item[] Preface\hfill ix

\vspace{-9pt}\item Computers, Objects, and Java\hfill 3
\begin{SEC}
\item Welcome\quad 4
\item Why Study Programming?\quad 4
\item Why Java?\quad 4
\item What Is a Computer?\quad 6
\item[]{\color{cyan}Processors Then and Now}\quad 8
\item The Internet and the World Wide Web\quad 8
\item Programming Languages\quad 10
\item What Is Object-Oriented Programming?\quad 12
\begin{SUBSEC}
\item Basic Object-Oriented Programming Metaphor: \\\hspace*{1pc}Interacting Objects\quad 12
\item What Is a Java Object?\quad 13
\item What Is a Java Class?\quad 14
\item What Is a Message?\quad 15
\item What Is a Class Hierarchy?\quad 15
\item What Is Class Inheritance?\quad 16
\item What Is an Interface?\quad 18
\item What Is Polymorphism?\quad 19
\item Principles of Object-Oriented Design\quad 20
\end{SUBSEC}
\item Summary of UML Elements\quad 22
\begin{SUBSEC}
\item Class Diagrams\quad 22
\item Packages\quad 23
\item Object Diagrams\quad 23
\item Collaboration Diagrams\quad 24
\item Sequence Diagrams\quad 24
\end{SUBSEC}
\item[]{\small Chapter Summary\quad 24

\vspace{-1pt}\item[] Exercises\quad 25}
\end{SEC}


\vspace{-9pt}\item Java Program Design and Development\hfill 29
\begin{SEC}
\item Introduction\quad 30
\item Designing Good Programs\quad 30
\begin{SUBSEC}
\item The Software Life Cycle\quad 30
\item Problem Decomposition\quad 32
\item Object Design\quad 33
\item Data, Methods, and Algorithms\quad 34
\item Coding into Java\quad 36
\item Testing, Debugging, and Revising\quad 38
\item[]{\color{cyan}Grace Hopper and the First Computer Bug}\quad 39
\item Writing Readable Programs\quad 39
\item[]{\color{cyan}Java Language Summary}\quad 40
\end{SUBSEC}
\item Editing, Compiling, and Running a Java Program\quad 45
\begin{SUBSEC}
\item Java Applications and Applets\quad 45
\item Java Development Environments\quad 46
\end{SUBSEC}
\item[]{\color{cyan}From the Java Library: System and Printstream}\quad 50
%���{\tt java.io.PrintStream}
\item Qualified Names in Java\quad 52
\item[]{\color{cyan}In the Laboratory: Editing, Compiling, \\\hspace*{1pc}and Running an Applet}\quad 53
\item[]{\small Chapter Summary\quad 60
\item[] Solutions to Self-Study Exercises\quad 61

\vspace{-1pt}\item[] Exercises\quad 62}
\end{SEC}

\vspace{-9pt}\item Objects: Defining, Creating, and Using\hfill 67
\begin{SEC}
\item Introduction\quad 68
\item Class Definition\quad 68
\begin{SUBSEC}
\item The {\tt Rectangle }Class\quad 68
\item The {\tt RectangleUser} Class\quad 70
\item Object Instantiation: Creating {\tt Rectangle} Instances\quad 70
\item Interacting with {\tt Rectangles}\quad 71
\item Define, Create, Use\quad 72
\end{SUBSEC}
\item{\color{cyan}Case Study: Simulating a CyberPet}\quad 72
\begin{SUBSEC}
\item Designing the {\tt CyberPet} Class\quad 73
\item Defining the {\tt CyberPet} Class\quad 74
\item Creating {\tt CyberPet} Instances\quad 82
\item Primitive Types and Reference Types\quad 84
\item Using {\tt CyberPets}\quad 85
\item Class Design: The {\tt TestCyberPet} Application\quad 86
\item Flow of Control: Method Call and Return\quad 87
\item Tracing the {\tt TestCyberPet} Program\quad 88
\item Class Design: The {\tt TestCyberPetApplet}\quad 90
\item Class Design: The {\tt CyberPet} Application\quad 91
\end{SUBSEC}
\item[]{\color{cyan}Object-Oriented Design: Basic Principles}\quad 91
\item[]{\color{cyan}Alan Kay and the Smalltalk Language}\quad 93
\item[]{\color{cyan}From the Java Library: {\tt BufferedReader}, {\tt String}, \\\hspace*{1pc}{\tt Integer}}\quad 93
\item[]{\color{cyan}In the Laboratory: The {\tt Circle} Class}\quad 98
\item[]{\color{cyan}Java Language Summary}\quad 100
\item[]{\small Chapter Summary\quad 106
\item[] Solutions to Self-Study Exercises\quad 107

\vspace{-1pt}\item[] Exercises\quad 109}
\end{SEC}

\vspace{-9pt}\item Methods: Communicating with Objects\hfill 115
\begin{SEC}
\item Introduction\quad 116
\item Passing Information to an Object\quad 116
\begin{SUBSEC}
\item Arguments and Parameters\quad 119
\item Passing a String to {\tt CyberPet}\quad 121
\item Parameters and the Generality Principle\quad 122
\end{SUBSEC}
\item Constructor Methods\quad 124
\begin{SUBSEC}
\item Default Constructors\quad 126
\item Constructor Overloading and Method Signatures\quad 126
\item Constructor Invocation\quad 128
\end{SUBSEC}
\item Retrieving Information from an Object\quad 128
\begin{SUBSEC}
\item Invoking a Method That Returns a Value\quad 129
\end{SUBSEC}
\item Passing a Value and Passing a Reference\quad 132
\item Flow of Control: Selection Control Structures\quad 134
\begin{SUBSEC}
\item The Simple If Statement\quad 135
\item The If/Else Statement\quad 138
\item The Nested If/Else Multiway Selection Structure\quad 138
\item The Dangling Else Problem\quad 140
\item The Switch Multiway Selection Structure\quad 142
\end{SUBSEC}
\item The Improved {\tt CyberPet}\quad 145
\item[]{\color{cyan}Intelligent Agents}\quad 148
\item[]{\color{cyan}From the Java Library: {\tt java.lang.Object}}\quad 148
\item[]{\color{cyan}Object-Oriented Design: Inheritance and \\\hspace*{1pc}Polymorphism}\quad 150
\item[]{\color{cyan}In the Laboratory: Feeding {\tt CyberPet}}\quad 151
\item[]{\color{cyan}Java Language Summary}\quad 153
\item[]{\small Chapter Summary\quad 156
\item[] Solutions to Self-Study Exercises\quad 156

\vspace{-1pt}\item[] Exercises\quad 159}
\end{SEC}

\item Applets: Programming for the World Wide Web\hfill 167

\begin{SEC}
\item Introduction\quad 168
\item The {\tt Applet} Class\quad 168
\begin{SUBSEC}
\item Java's GUI Components\quad 168
\end{SUBSEC}
\item Class Inheritance\quad 170
\begin{SUBSEC}
\item Objects, Assignments, and Types (Optional)\quad 171
\item Defining a {\tt Square} as a Subclass \\\hspace*{1pc}of {\tt Rectangle} (Optional)\quad 172
\end{SUBSEC}
\item Applet Subclasses\quad 174
\item A Simple Applet\quad 175
\begin{SUBSEC}
\item Inheriting Functionality\quad 176
\item Implementing an Interface\quad 178
\item Extending Functionality\quad 179
\item Using the Inheritance Hierarchy\quad 180
\item Polymorphism and Extensibility (Optional)\quad 182
\end{SUBSEC}
\item Event-Driven Programming\quad 184
\begin{SUBSEC}
\item The Java Event Model\quad 185
\item Tracing an Applet\quad 188
\end{SUBSEC}
\item{\color{cyan}Case Study: The {\tt CyberPetApplet}}\quad 190
\begin{SUBSEC}
\item Specifying the Interface\quad 190
\item Designing CyberPetApplet\quad 191
\item Applet Layout\quad 195
\item Handling CyberPetApplet Actions\quad 197
\item Running CyberPetApplet\quad 198
\end{SUBSEC}
\item[]{\color{cyan}Object-Oriented Design: Inheritance and \\\hspace*{1pc}Polymorphism (Optional)}\quad 198
\item[]{\color{cyan}Tim Berners-Lee, Creator of the WWW}\quad 200
\item[]{\color{cyan}From the Java Library: {\tt java.awt.Image}}\quad 204
\item[]{\color{cyan}In the Laboratory: {\tt CyberPetApplet}}\quad 205
\item[]{\color{cyan}Java Language Summary}\quad 209
\item[]{\small Chapter Summary\quad 211
\item[] Solutions to Self-Study Exercises\quad 212

\vspace{-1pt}\item[] Exercises\quad 213}
\end{SEC}

\vspace{-8pt}\item Java Data and Operators\hfill 219

\begin{SEC}
\item Introduction\quad 220
\item Programming = Representation + Action\quad 220
\item Boolean Data and Operators\quad 221
\begin{SUBSEC}
\item Boolean (or Logical) Operations\quad 222
\item Precedence, Associativity, and Commutativity\quad 222
\end{SUBSEC}
\item The Boolean-Based {\tt CyberPet} Model\quad 224
\item[]{\color{cyan}Are We Computers?}\quad 226
\item Numeric Data and Operators\quad 227
\begin{SUBSEC}
\item Numeric Operations\quad 228
\item Operator Precedence\quad 231
\item Increment and Decrement Operators\quad 232
\item Assignment Operators\quad 234
\item Relational Operators\quad 235
\end{SUBSEC}
\item{\color{cyan}Case Study: Converting Fahrenheit to Celsius}\quad 237
\begin{SUBSEC}
\item Problem Decomposition\quad 237
\item Class Design: {\tt Temperature}\quad 237
\item Testing and Debugging\quad 238
\item The {\tt TemperatureTest} Class\quad 239
\item Algorithm Design: Data Conversion\quad 240
\item The {\tt TemperatureApplet} Class\quad 241
\end{SUBSEC}
\item An Integer-Based {\tt CyberPet} Model\quad 245
\begin{SUBSEC}
\item Class Constants\quad 245
\item The Revised {\tt CyberPet} Class\quad 248
\item Advantages of the Integer-Based {\tt CyberPet}\quad 249
\end{SUBSEC}
\item[]{\color{cyan}Object-Oriented Design: Information Hiding}\quad 250
\item Character Data and Operators\quad 251
\begin{SUBSEC}
\item Character to Integer Conversions\quad 252
\item Lexical Ordering\quad 254
\item Relational Operators\quad 254
\end{SUBSEC}
\item Example: Character Conversions\quad 254
\begin{SUBSEC}
\item Static Methods\quad 256
\end{SUBSEC}
\item[]{\color{cyan}From the Java Library: {\tt java.lang.Math}}\quad 257
\item[]{\color{cyan}From the Java Library: {\tt java.text.NumberFormat}}\quad 259
\item Example: Calculating Compound Interest\quad 260
\item Problem Solving = Representation + Action\quad 263
\item[]{\color{cyan}In the Laboratory: The Leap-Year Problem}\quad 263
\item[]{\color{cyan}Java Language Summary}\quad 266
\item[]{\small Chapter Summary\quad 268

\item[] Solutions to Self-Study Exercises\quad 269


\vspace{-1pt}\item[] Exercises\quad 270}
\end{SEC}

\vspace{-10pt}\item Control Structures\hfill 275

\begin{SEC}
\item Introduction\quad 276

\item Flow of Control: Repetition Structures\quad 276
\item Counting Loops\quad 278
\begin{SUBSEC}
\item The For Structure\quad 278
\item Loop Bounds\quad 280
\item Infinite Loops\quad 280
\item Loop Indentation\quad 281
\item Nested Loops\quad 283
\end{SUBSEC}
\item Example: Car Loan\quad 285
\item Conditional Loops\quad 287
\begin{SUBSEC}
\item The While Structure\quad 287
\item The Do-While Structure\quad 289
\end{SUBSEC}
\item Example: Computing Averages\quad 292
\item Example: Data Validation\quad 296
\item{\color{cyan}Case Study: Animated {\tt CyberPet}}\quad 299
\begin{SUBSEC}
\item Problem Description and Specification\quad 299
\item Class Design: {\tt CyberPetApplet}\quad 299
\item Algorithm Design: {\tt doEatAnimation()}\quad 300
\item Implementation\quad 301
\end{SUBSEC}
\item Principles of Loop Design\quad 303
\item[]{\color{cyan}Object-Oriented Design: Structured Programming}\quad 304
\item[]{\color{cyan}What Can Be Computed?}\quad 310
\item[]{\color{cyan}From the Java Library: {\tt java.awt.TextArea}}\quad 311
\item[]{\color{cyan}Cryptography}\quad 314
\item[]{\color{cyan}In the Laboratory: Finding Prime Numbers}\quad 314
\item[]{\color{cyan}Java Language Summary}\quad 319
\item[]{\small Chapter Summary\quad 320
\item[] Solutions to Self-Study Exercises\quad 321

\vspace{-1pt}\item[] Exercises}\quad 325
\end{SEC}

\vspace{-10pt}\item Strings and String Processing\hfill 331

\begin{SEC}
\item Introduction\quad 332
\item String Basics\quad 332
\begin{SUBSEC}
\item Constructing Strings\quad 333
\item Concatenating Strings\quad 334
\item Indexing Strings\quad 335
\item Converting Data to String\quad 336
\end{SUBSEC}
\item Finding Things Within a String\quad 337
\item Example: Keyword Search\quad 338
\item[]{\color{cyan}From the Java Library: {\tt java.lang.StringBuffer}}\quad 340
\item Retrieving Parts of Strings\quad 342
\item Example: Processing Names and Passwords\quad 343
\item Processing Each Character in a String\quad 345
\begin{SUBSEC}
\item Off-by-One Error\quad 345
\item Example: Counting Characters\quad 346
\end{SUBSEC}
\item{\color{cyan}Case Study: Silly {\tt CyberPet} String Tricks}\quad 346
\begin{SUBSEC}
\item Class Design: {\tt StringTricks}\quad 346
\item Method Design: {\tt getNextTrick()}\quad 347
\item Method Design: {\tt reverse()}\quad 347
\item Method Design: {\tt toUpperCase()}\quad 348
\item Method Design: {\tt capitalize()}\quad 349
\item Miscellaneous {\tt String} Methods\quad 349
\end{SUBSEC}
\item Comparing Strings\quad 350
\begin{SUBSEC}
\item Object Identity Versus Object Equality\quad 352
\item String Identity Versus String Equality\quad 354
\end{SUBSEC}
\item[]{\color{cyan}From the Java Library: {\tt java.util.StringTokenizer}}\quad 357
\item[]{\color{cyan}Historical Cryptography}\quad 358
\item[]{\color{cyan}Object-Oriented Design: The Abstract Cipher Class}\quad 359
\item[]{\color{cyan}In the Laboratory: Pig Latin Translation}\quad 365
\item[]{\color{cyan}Java Language Summary}\quad 368
\item[]{\small Chapter Summary\quad 368
\item[] Solutions to Self-Study Exercises\quad 369

\vspace{-1pt}\item[] Exercises\quad 372}
\end{SEC}
\item Arrays and Array Processing\hfill 377
\begin{SEC}
\item Introduction\quad 378
\item One-Dimensional Arrays\quad 379
\begin{SUBSEC}
\item Declaring and Creating Arrays\quad 380
\item Array Allocation\quad 381
\item Initializing Arrays\quad 383
\item Assigning and Using Array Values\quad 384
\end{SUBSEC}
\item Simple Array Examples\quad 384
\item Example: Testing a Die\quad 386
\begin{SUBSEC}
\item Generating Random Numbers\quad 387
\item The Die-Testing Experiment\quad 388
\end{SUBSEC}
\item{\color{cyan}Case Study: CyberPet Animation}\quad 392
\item[]{\color{cyan}Data Compression}\quad 394
\item Array Algorithms: Sorting\quad 395
\begin{SUBSEC}
\item Bubble Sort\quad 395
\item Algorithm: Swapping Memory Elements\quad 397
\item Selection Sort\quad 399
\item Passing Array Parameters\quad 400
\end{SUBSEC}
\item Array Algorithms: Searching\quad 402
\begin{SUBSEC}
\item Sequential Search\quad 402
\item Binary Search\quad 404
\end{SUBSEC}
\item Two-Dimensional Arrays\quad 407
\begin{SUBSEC}
\item Two-Dimensional Array Methods\quad 409
\item Passing Part of an Array to a Method\quad 411
\end{SUBSEC}
\item Multidimensional Arrays\quad 415
\begin{SUBSEC}
\item Array Initializers\quad 416
\item[]{\color{cyan}Object-Oriented Design: Polymorphic \\\hspace*{1pc}Sorting (Optional)}\quad 416
\end{SUBSEC}
\item[]{\color{cyan}From the Java Library: {\tt java.util.Vector}}\quad 419
\item{\color{cyan}Case Study: Simulating a Card Deck}\quad 420
\begin{SUBSEC}
\item Designing a {\tt Card }class\quad 420
\item Designing a {\tt Deck }class\quad 423
\end{SUBSEC}
\item[]{\color{cyan}In the Laboratory: A Card-Game Applet}\quad 425
\item[]{\color{cyan}Java Language Summary}\quad 428
\item[]{\small Chapter Summary\quad 429
\item[] Solutions to Self-Study Exercises\quad 429

\vspace{-1pt}\item[] Exercises\quad 433}
\end{SEC}

\item Graphical User Interfaces\hfill 439

\vspace{6pt}\begin{SEC}
\item Introduction\quad 440
\item[]{\color{cyan}From the Java Library: AWT to Swing}\quad 440
\item The Swing Component Set\quad 444
\item[]{\color{cyan}Object-Oriented Design: \\\hspace*{1pc}Model-View-Controller Architecture}\quad 445
\item The Java Event Model\quad 446
\begin{SUBSEC}
\item Events and Listeners\quad 447
\item Event Classes\quad 448
\end{SUBSEC}
\item{\color{cyan}Case Study: Designing a Basic GUI}\quad 450
\begin{SUBSEC}
\item The Metric Converter Application\quad 451
\item Inner Classes and Adapter Classes\quad 455
\item GUI Design Critique\quad 456
\item Extending the Basic GUI: Button Array\quad 457
\item GUI Design Critique\quad 460
\end{SUBSEC}
\item Containers and Layout Managers\quad 463
\begin{SUBSEC}
\item Layout Managers\quad 463
\item The {\tt GridLayout }Manager\quad 464
\item GUI Design Critique\quad 465
\item The {\tt BorderLayout} Manager\quad 465
\end{SUBSEC}
\item Checkboxes, Radio Buttons, and Borders\quad 467
\begin{SUBSEC}
\item Checkbox and Radio Button Arrays\quad 469
\item Swing Borders\quad 470
\item The {\tt BoxLayout} Manager\quad 471
\item The {\tt ItemListener} Interface\quad 471
\item The {\tt OrderApplet}\quad 472
\end{SUBSEC}
\item Menus and Scroll Panes\quad 475
\begin{SUBSEC}
\item Adding a Menu Bar to an Application\quad 475
\item Menu Hierarchies\quad 477
\item Handling Menu Actions\quad 478
\item Adding Scrollbars to a Text Area\quad 479
\end{SUBSEC}
\item[]{\color{cyan}In the Laboratory: The ATM Machine}\quad 482
\item[]{\color{cyan}Are Computers Intelligent?}\quad 483
\item[]{\small Chapter Summary\quad 489
\item[] Solutions to Self-Study Exercises\quad 490

\vspace{-1pt}\item[] Exercises\quad 492}
\end{SEC}
\item Graphics and Drawing\hfill 495
\begin{SEC}
\item Introduction\quad 496
\item The Drawing Surface\quad 496
\item The Graphics Context\quad 496
\begin{SUBSEC}
\item Graphics Color and Component Color\quad 498
\item The Graphics Coordinate System\quad 499
\item Properties of the Graphics Context\quad 499
\end{SUBSEC}
\item The {\tt Color }Class\quad 500
\begin{SUBSEC}
\item Example: The {\tt ColorPicker }Applet\quad 502
\item Painting Components\quad 504
\end{SUBSEC}
\item[]{\color{cyan}Object-Oriented Design: Reference Constants}\quad 506
\item[]{\color{cyan}From the Java Library: {\tt Point}s and {\tt Dimension}s}\quad 507
\item Painting and Drawing Lines and Shapes\quad 508
\item Example: The {\tt ShapeDemo} Applet\quad 509
\item Graphing Equations\quad 512
\begin{SUBSEC}
\item Example: The {\tt Graph }Program\quad 513
\end{SUBSEC}
\item Drawing Bar Charts and Pie Charts\quad 517
\begin{SUBSEC}
\item Scaling the Bar Chart\quad 517
\item Drawing Arcs\quad 520
\end{SUBSEC}
\item Handling Text in a Graphics Context\quad 523
\begin{SUBSEC}
\item The {\tt Font }and {\tt FontMetrics }Classes\quad 523
\item Example: The {\tt FontNames }Applet\quad 524
\item Font Metrics\quad 525
\item Example: Centering a Line of Text\quad 526
\end{SUBSEC}
\item{\color{cyan}Case Study: Interactive Drawing}\quad 528
\begin{SUBSEC}
\item Handling Mouse Events\quad 528
\item An Interactive Painting Program\quad 528
\item GUI Design\quad 528
\item Problem Decomposition: The MousePaint Class\quad 529
\item Algorithm: Handling Mouse Events\quad 529
\item Algorithm: The {\tt paintComponent()} Method\quad 530
\end{SUBSEC}
\item[]{\color{cyan}Object-Oriented Design: The Scalable {\tt CyberPet}}\quad 533
\item[]{\color{cyan}In the Laboratory: The {\tt SelfPortrait} Class}\quad 541
\item[]{\small Chapter Summary\quad 543
\item[] Solutions to Self-Study Exercises\quad 544

\vspace{-1pt}\item[] Exercises\quad 547}
\end{SEC}

\item Exceptions: When Things Go Wrong\hfill 551

\begin{SEC}
\item Introduction\quad 552
\item Handling Exceptional Conditions\quad 552
\begin{SUBSEC}
\item Traditional Error Handling\quad 553
\item Java's Default Exception Handling\quad 553
\end{SUBSEC}
\item Java's Exception Hierarchy\quad 554
\begin{SUBSEC}
\item Checked and Unchecked Exceptions\quad 556
\item The {\tt Exception }Class\quad 557
\end{SUBSEC}
\item Handling Exceptions Within a Program\quad 558
\begin{SUBSEC}
\item Trying, Throwing, and Catching an Exception\quad 558
\item Separating Error Checking from Error Handling\quad 560
\item Syntax and Semantics of Try/Throw/Catch\quad 561
\item Restrictions on the Try/Catch/Finally Statement\quad 563
\item Dynamic Versus Static Scoping\quad 564
\item Exception Propagation: Searching for a Catch Block\quad 565
\end{SUBSEC}
\item Error Handling and Robust Program Design\quad 568
\begin{SUBSEC}
\item Print a Message and Terminate\quad 569
\item Log the Error and Resume\quad 569
\item Fix the Error and Resume\quad 570
\item To Fix or Not to Fix\quad 573
\end{SUBSEC}
\item Creating and Throwing Your Own Exceptions\quad 576
\item[]{\color{cyan}From the Java Library: {\tt javax.swing.JOptionPane}}\quad 580
\item[]{\color{cyan}In the Laboratory: Measuring Exception Overhead}\quad 581
\item[]{\color{cyan}Java Language Summary}\quad 586
\item[]{\small Chapter Summary\quad 587
\item[] Solutions to Self-Study Exercises\quad 588

\vspace{-1pt}\item[] Exercises\quad 590}
\end{SEC}

\vspace{-9pt}\item Recursive Problem Solving\hfill 595

\begin{SEC}
\item Introduction\quad 596
\begin{SUBSEC}
\item Recursion as Repetition\quad 596
\item Recursion as a Problem-Solving Approach\quad 598
\end{SUBSEC}
\item Recursive Definition\quad 598
\begin{SUBSEC}
\item Factorial: {\it N}!\quad 598
\item Drawing a Nested Pattern\quad 599
\end{SUBSEC}
\item Recursive String Methods\quad 600
\begin{SUBSEC}
\item Printing a String\quad 601
\item Printing the String Backward\quad 604
\item Counting Characters in a String\quad 605
\item Translating a String\quad 607
\end{SUBSEC}
\item Recursive Array Processing\quad 609
\begin{SUBSEC}
\item Recursive Sequential Search\quad 609
\item Information Hiding\quad 611
\item Recursive Selection Sort\quad 613
\end{SUBSEC}
\item Example: Drawing (Recursive) Fractals\quad 615
\begin{SUBSEC}
\item Nested Squares\quad 615
\item The Sierpinski Gasket\quad 617
\end{SUBSEC}
\item[]{\color{cyan}Object-Oriented Design: Tail Recursion}\quad 618
\item[]{\color{cyan}Object-Oriented Design: Recursion or Iteration?}\quad 620
\item[]{\color{cyan}Exploring the Mandelbrot Set}\quad 621
\item[]{\color{cyan}From the Java Library: {\tt javax.swing.JComboBox}}\quad 622
\item[]{\color{cyan}In the Laboratory: The {\tt RecursivePatterns} Applet}\quad 627
\item[]{\small Chapter Summary\quad 628
\item[] Solutions to Self-Study Exercises\quad 628

\vspace{-1pt}\item[] Exercises\quad 631}
\end{SEC}

\vspace{-9pt}\item Threads and Concurrent Programming\hfill 635
\begin{SEC}
\item Introduction\quad 636
\item What Is a Thread?\quad 636
\begin{SUBSEC}
\item Concurrent Execution of Threads\quad 637
\item Multithreaded Numbers\quad 638
\end{SUBSEC}
\item[]{\color{cyan}From the Java Library: {\tt java.lang.Thread}}\quad 640
\item Thread States and Life Cycle\quad 645
\item Using Threads to Improve Interface Responsiveness\quad 647
\begin{SUBSEC}
\item Single-Threaded Design\quad 647
\item Multithreaded Drawing: The Dotty Thread\quad 651
\item Advantages of Multithreaded Design\quad 654
\end{SUBSEC}
\item{\color{cyan}Case Study: Cooperating Threads}\quad 655
\begin{SUBSEC}
\item Problem Statement\quad 656
\item Design: The {\tt TakeANumber} Class\quad 656
\item Java Monitors and Mutual Exclusion\quad 658
\item The {\tt Customer }Class\quad 659
\item The {\tt Clerk }Class\quad 660
\item The {\tt Bakery }Class\quad 660
\item Problem: Critical Sections\quad 662
\item Using Wait/Notify to Coordinate Threads\quad 666
\end{SUBSEC}
\item{\color{cyan}Case Study: The Spider and Fly Threads}\quad 669
\begin{SUBSEC}
\item Problem Decomposition\quad 670
\item The Revised {\tt CyberPet }Class\quad 671
\item The Fly Thread\quad 673
\item The Spider Thread\quad 676
\item The {\tt CyberPetApplet }Class\quad 680
\end{SUBSEC}
\item[]{\color{cyan}Object-Oriented Design: Inheritance and \\\hspace*{1pc}Polymorphism}\quad 680
\item[]{\color{cyan}In the Laboratory: The Spider, the Fly, and the Bee}\quad 682
\item[]{\small Chapter Summary\quad 685
\item[] Solutions to Self-Study Exercises\quad 687

\vspace{-1pt}\item[] Exercises\quad 688}
\end{SEC}

\vspace{-9pt}\item Files, Streams, and Input/Output Techniques\hfill 691

\vspace{-4pt}\begin{SEC}
\item Introduction\quad 692
\item Streams and Files\quad 692
\begin{SUBSEC}
\item The Data Hierarchy\quad 693
\item Binary Files and Text Files\quad 693
\item Input and Output Streams\quad 694
\end{SUBSEC}
\item{\color{cyan}Case Study: Reading and Writing Text Files}\quad 698
\begin{SUBSEC}
\item Text File Format\quad 698
\item Writing to a Text File\quad 698
\item Code Reuse: Designing Text File Output\quad 700
\item Reading from a Text File\quad 702
\item Code Reuse: Designing Text File Input\quad 706
\item The {\tt TextIO }Application\quad 707
\end{SUBSEC}
\item The {\tt File }Class\quad 709
\begin{SUBSEC}
\item Names and Paths\quad 709
\item Validating File Names\quad 710
\end{SUBSEC}
\item Example: Reading and Writing Binary Files\quad 711
\begin{SUBSEC}
\item Writing Binary Data\quad 712
\item Reading Binary Data\quad 715
\item The {\tt BinaryIO} Application\quad 717
\item Abstracting Data from Files\quad 717
\end{SUBSEC}
\item Object Serialization: Reading and Writing Objects\quad 720
\begin{SUBSEC}
\item The {\tt ObjectIO} Class\quad 722
\end{SUBSEC}
\item[]{\color{cyan}From the Java Library: {\tt javax.swing.JFileChooser}}\quad 724
\item[]{\color{cyan}In the Laboratory: The {\tt TextEdit} Program}\quad 726
\item[]{\color{cyan}Databases and Personal Privacy}\quad 727
\item[]{\small Chapter Summary\quad 729
\item[] Solutions to Self-Study Exercises\quad 731

\vspace{-1pt}\item[] Exercises\quad 732}
\end{SEC}

\vspace{-11pt}\item Sockets and Networking\hfill 737

\vspace{-3pt}\begin{SEC}
\item Introduction\quad 738
\item An Overview of Networks\quad 738
\begin{SUBSEC}
\item Network Size and Topology\quad 738
\item Internets\quad 739
\item Network Protocols\quad 740
\item Client/Server Applications\quad 741
\item Lower Level Network Protocols\quad 742
\item The {\tt java.net} Package\quad 743
\end{SUBSEC}
\item Using Network Resources from an Applet\quad 744
\item[]{\color{cyan}From the Java Library: {\tt java.net.URL}}\quad 745
\item The Slide Show Applet\quad 746
\begin{SUBSEC}
\item The {\tt SlideShowApplet} Class\quad 747
\item The {\tt Timer} Class\quad 750
\end{SUBSEC}
\item Using Network Resources from an Application\quad 751
\begin{SUBSEC}
\item Downloading a Text File from the Web\quad 751
\item Code Reuse: The {\tt java.awt.Toolkit} Class\quad 753
\item The {\tt RealEstateViewer} Application\quad 754
\item Reusing Code\quad 758
\end{SUBSEC}
\item Client/Server Communication via Sockets\quad 761
\begin{SUBSEC}
\item The Server Protocol\quad 762
\item The Client Protocol\quad 763
\item A Two-Way Stream Connection\quad 763
\end{SUBSEC}
\item{\color{cyan}Case Study: Generic Client/Server Classes}\quad 765
\begin{SUBSEC}
\item Object-Oriented Design\quad 766
\item The {\tt EchoServer} Class\quad 768
\item The {\tt EchoClient} Class\quad 771
\item Abstracting the Generic Server\quad 774
\item Testing the Echo Service\quad 779
\end{SUBSEC}
\item Java Network Security Restrictions\quad 780
\item[]{\color{cyan}In the Laboratory: The Internet CyberPet}\quad 781
\item[]{\color{cyan}Privacy and the Internet}\quad 782
\item[]{\small Chapter Summary\quad 786
\item[] Solutions to Self-Study Exercises\quad 787

\vspace{-1pt}\item[] Exercises\quad 788}
\end{SEC}

\vspace{-11pt}\item Data Structures: Lists, Stacks, and Queues\hfill 791

\vspace{-3pt}\begin{SEC}
\item Introduction\quad 792
\item The Linked List Data Structure\quad 792
\begin{SUBSEC}
\item Using References to Link Objects\quad 792
\item Example: The Dynamic Phone List\quad 794
\item Manipulating the Phone List\quad 796
\end{SUBSEC}
\item[]{\color{cyan}Object-Oriented Design: The List Abstract \\\hspace*{1pc}Data Type (ADT)}\quad 803
\item The Stack ADT\quad 808
\begin{SUBSEC}
\item The {\tt Stack }Class\quad 809
\item Testing the {\tt Stack} Class\quad 810
\end{SUBSEC}
\item The Queue ADT\quad 811
\begin{SUBSEC}
\item The {\tt Queue} Class\quad 811
\end{SUBSEC}
\item[]{\color{cyan}From the Java Library: {\tt java.util.Stack}}\quad 812
\item[]{\color{cyan}In the Laboratory: Capital Gains}\quad 813
\item[]{\color{cyan}The LISP Language}\quad 814
\item[]{\small Chapter Summary\quad 818
\item[] Solutions to Self-Study Exercises\quad 819

\vspace{-1pt}\item[] Exercises\quad 822}
\end{SEC}
\end{CH}
\begin{APP}
\item Coding Conventions\hfill 824
\begin{APPSEC}
\item Comments\quad 824
\item Indentation and White Space\quad 826
\item Naming Conventions\quad 826
\item Use of Braces\quad 827
\item File Names and Layout\quad 828
\item Statements\quad 828
\item Executable Statements\quad 830
\item Preconditions and Postconditions\quad 830
\item Sample Programs\quad 830
\end{APPSEC}
\item The Java Development Kit\hfill 831
\begin{APPSEC}
\item The Java Compiler: {\tt javac}\quad 831
\item The Java Interpreter: {\tt java}\quad 833
\item The {\tt appletviewer}\quad 833
\item The Java Archiver {\tt jar} Tool\quad 837
\item The Java Documentation Tool: {\tt javadoc}\quad 838
\end{APPSEC}
\item The ASCII and Unicode Character Sets\hfill 839
\item Java Keywords\hfill 840
\item Operator Precedence Hierarchy\hfill 841
\item Advanced Language Features\hfill 843
\begin{APPSEC}
\item Inner Classes\quad 843
\item Nested Top-Level Versus Member Classes\quad 844
\item Local and Anonymous Inner Classes\quad 845
\end{APPSEC}
\item Java and UML Resources\hfill 849
\begin{APPSEC}
\item Reference Books\quad 849
\item Online References\quad 849
\end{APPSEC}

\vspace{12pt}\item[] Subject Index\hfill 851
\end{APP}
%
\end{document}
