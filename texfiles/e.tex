\setcounter{table}{0}
\setcounter{figure}{0}
\renewcommand{\thetable}{\mbox{E.\arabic{table}}}%
\renewcommand{\thefigure}{\mbox{E--\arabic{figure}}}%


\chapter{Operator Precedence Hierarchy}

\markboth{{\color{cyan}APPENDIX\,E\,\,$\bullet$\,\,}Operator Precedence Hierarchy}
{{\color{cyan}APPENDIX\,E\,\,$\bullet$\,\,}Operator Precedence Hierarchy}

%%\appendixright{E}{Operator Precedence Hierarchy}

\noindent Table E.1 summarizes the precedence and
associativity relationships for Java operators.   Within a single
expression, an operator of order {\it m} would be evaluated before an
operator of order {\it n} if $m < n$. Operators having the same order
are evaluated according to their association order.  For example, the
expression

\begin{jjjlisting}
\begin{lstlisting}
25 + 5 * 2 + 3
\end{lstlisting}
\end{jjjlisting}

\noindent would be evaluated in the order shown by the following
parenthesized expression:

\begin{jjjlisting}
\begin{lstlisting}
(25 + (5 * 2)) + 3   ==> (25 + 10) + 3 ==> 35 + 3  ==> 38
\end{lstlisting}
\end{jjjlisting}

\noindent In other words, because \verb|*| has higher precedence
than \verb|+|, the multiplication operation is done before either of
the addition operations.   And because addition associates from left to
right, addition operations are performed from left to right.



Most operators associate from left to right, but note that assignment
operators associate from right to left.   For example, consider the
following code segment:

\begin{jjjlisting}
\begin{lstlisting}
int i, j, k;
i = j = k = 100;     // Equivalent to i = (j = (k = 100));
\end{lstlisting}
\end{jjjlisting}

\begin{table}[h]
%\hphantom{\caption{Java operator precedence and associativity table.\index{precedence table}}}
\TBT{0pc}{Java operator precedence and associativity table.}
\hspace*{-6pt}\begin{tabular}{clll}
\multicolumn{4}{l}{\color{cyan}\rule{29pc}{1pt}}\\[2pt]
%%%%\TBCH{{\bf Order}} & \TBCH{{\bf Operator}} & \TBCH{{\bf Operation}} & \TBCH{{\bf Association}}
{\bf Order} & {\bf Operator} & {\bf Operation} & {\bf Association}
\\[-4pt]\multicolumn{4}{l}{\color{cyan}\rule{29pc}{0.5pt}}\\[2pt]
0 &\verb|(  )|&{\it Parentheses}\cr
1 &\verb|++   -- |$\;\cdot$&{\it Postincrement,  Postdecrement, Dot Operator}&{\it L to R}\cr
2 &\verb|++   --  +  -  !|&{\it Preincrement,  Predecrement, }&{\it R to L}\cr
  &&{\it Unary plus,  Unary minus,  Boolean NOT}&\cr
3 &\verb|(type)  new|&{\it Type  Cast, Object Instantiation}&{\it R to L}\cr
4 &\verb|*  /  %|&{\it Multiplication,  Division,  Modulus}&{\it L to R}\cr
5 &\verb|+ -  +|&{\it Addition,  Subtraction,  String Concatenation}&{\it L to R}\cr
6 &\verb|<  >  <=  >=|&{\it Relational  Operators}&{\it L to R}\cr
7 &\verb|==   !=|&{\it Equality  Operators}&{\it L to R}\cr
8 &$\wedge$&{\it Boolean  XOR}&{\it L to R}\cr
9 &\verb|&&|&{\it Boolean  AND}&{\it L to R}\cr
10&\verb||||&{\it Boolean  OR}&{\it L to R}\cr
11&\verb|= += -= *= /= %=|&{\it Assignment  Operators}&{\it R to L}
\\[-4pt]\multicolumn{4}{l}{\color{cyan}\rule{29pc}{1pt}}
\end{tabular}
\endTB
\end{table}

\noindent  In this case, each variable will be assigned 100 as its
value.  But it's important that this expression be evaluated from right
to left.  First, {\it k} is assigned 100. Then its value is assigned
to {\it j}. And finally {\it j}'s value is assigned to {\it i}.

For expressions containing mixed operators, it's always a good idea to
use parentheses to clarify the order of evaluation.  This will also
help avoid subtle syntax and semantic errors.
